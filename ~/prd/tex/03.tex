\section{Introduction}\label{introduction}

A discipline is established as a categorization mechanism when the set
of possible labels for the same cultural object shrinks to a short and
ranked list.

The position at the top of that list will be occupied by what we call
the disciplinary prefix, and below it variations on nouns used to
capture different aspects of cultrual content from created work to the
authors themselves.

Consider five social science disciplines--anthropology, economics,
political science, psychology, and sociology.

In english the labels that like flags lay claim to disciplinary
resources are the prefixes anth-, econ-, poli-, psyc-, and soci-.

In sociology the term ``social problem'' is an example of the flag being
established as a claim to disciplinary relevance.

We contend that these prefixes will diffuse first as generic and weakly
categorical terms that could modify and lay claim to any worldly object.

\section{Temporal Sequencing Methods}\label{temporal-sequencing-methods}

Correlations between time series are difficult to tease out due to
several dynamics that if not controlled for can lead to spurious
correlations. Before we can attempt to test causal order we must
decompose historical trends in terms into their systematic and residual
components, such that we may test the residuals for patterns between two
series.

ARIMA models have been criticized for their irrealism (Isaac and Griffin
1989:877). After establishing statistical considerations and laying bare
our assumptions, we will discuss the historical and ontological
limitations of the statistical approach.

\subsection{Series}\label{series}

\begin{table}[!htbp] \centering 
  \caption{Terms searched in the Google Books Ngrams Database} 
  \label{query} 
\begin{tabular}{@{\extracolsep{5pt}} llllll} 
\\[-1.8ex]\hline 
\hline \\[-1.8ex] 
& soci & econ & anth & poli & psyc \\ 
\hline \\[-1.8ex] 
Genre & social & economic & cultural & political & mental \\ 
Technique & sociological & economical & anthropological & political & psychological \\ 
Ontology & society & economy & culture & polity & mind \\ 
Discipline & sociology & economics & anthropology & political science  & psychology \\ 
Profession & sociologist & economist & anthropologist & political scientist & pscyhologist \\ 
Subdiscipline & sociology of & economics of & anthropology of & political science of & psychology of \\ 
\hline \\[-1.8ex] 
\end{tabular} 
\end{table}

\subsection{ARIMA model}\label{arima-model}

ARIMA, or AutoRegressive Integrated Moving Average, models are effective
in decomposing several categories of within-series correlations.

\begin{equation}
\text{I} = \frac{\text{MA}}{\text{AR}}
\end{equation}

This says that \(I\), the change in our series, is a function of \(MA\),
a moving but systematic average (a line or higher order polynomial) and

\begin{equation}
 (1-B)^d y_{t} = \frac{c + (1 + \theta_1 B + \cdots + \theta_q B^q)e_t}{(1-\phi_1B - \cdots - \phi_p B^p)}
\end{equation}

Where \(c\) is a constant drift up or down,

\subsection{Granger Causality}\label{granger-causality}

\section{Results}\label{results}

As table \ref{t-prefix} shows.

\section{Which came first?}\label{which-came-first}

Granger tests can help determine which (Thurman and Fisher 1988; Granger
1969)

Clear secular trends and period effects surrounding WWII and the baby
boom. To control:

\begin{itemize}
\tightlist
\item
  Model the trends. We could estimate the linear trend or splines and
  then subtract them.
\item
  First differences. Subtract from each point the previous point.
\item
  Link relatives. Divide each point from the point before it.
\end{itemize}

Box Cox doesn't mean

\begin{equation}\tag{8.1}\label{eq-8-arima} y'_{t} = c +
\phi_{1}y'_{t-1} + \cdots + \phi_{p}y'_{t-p} +
\theta_{1}e_{t-1} + \cdots + \theta_{q}e_{t-q} + e_{t},
\end{equation}

\section*{References}\label{references}
\addcontentsline{toc}{section}{References}

\hypertarget{refs}{}
\hypertarget{ref-Granger:1969wx}{}
Granger, C W J. 1969. ``Investigating Causal Relations by Econometric
Models and Cross-spectral Methods.'' \emph{Econometrica} 37 (3): 424.

\hypertarget{ref-Isaac:1989hp}{}
Isaac, Larry W, and Larry J Griffin. 1989. ``Ahistoricism in Time-Series
Analyses of Historical Process: Critique, Redirection, and Illustrations
from U.S. Labor History.'' \emph{American Sociological Review} 54 (6):
873.

\hypertarget{ref-Thurman:1988va}{}
Thurman, Walter N, and Mark E Fisher. 1988. ``Chickens, Eggs, and
Causality, or Which Came First?'' \emph{American Journal of Agricultural
Economics} 70 (2): 237.
